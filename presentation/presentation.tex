\documentclass{beamer}

\usetheme{dtyped}

\usepackage{hyperref}
\hypersetup{
  colorlinks=true,
  allcolors=.,
  urlcolor=blue,
}
\usepackage[
  backend=biber,
  style=numeric,
  isbn=false,
  url=false,
  doi=false
]{biblatex}
\addbibresource{citations.bib}

% from https://tex.stackexchange.com/questions/178800/creating-sections-each-with-title-pages-in-beamers-slides
\AtBeginSection[]{
  \begin{frame}
  \vfill
  \centering
\begin{beamercolorbox}[sep=8pt,center,shadow=true,rounded=true]{title}
    \usebeamerfont{title}\insertsectionhead\par%
  \end{beamercolorbox}
  \vfill
  \end{frame}
}

\title{\texttt{dependently-typed}}
\subtitle{A programming languages club at Georgia Tech}
\author{Elton Pinto}
\date{\today}

\begin{document}

\begin{frame}
  \titlepage
\end{frame}

\begin{frame}
  \begin{center}
    Slides and proposal can be found at \\ \small\url{https://github.com/dependently-typed/coc-proposal}
  \end{center}
\end{frame}

\begin{frame}{Outline}
  \tableofcontents[hideallsubsections]
\end{frame}

\section{What is \texttt{dependently-typed}?}

\begin{frame}{About the club}
  \begin{columns}
    \begin{column}{0.8\textwidth}
      \begin{itemize}
      \item Programming languages (PL) and compilers club
      \item Open to undergraduate, graduate, and PhD students
      \item Named after Richard Eisenberg's \citetitle{eisenberg2016dependent} \cite{eisenberg2016dependent}
      \end{itemize}
    \end{column}

    \begin{column}{0.15\textwidth}
      \centering
      \includegraphics[width=\textwidth]{assets/dragon.png}
    \end{column}
  \end{columns}
\end{frame}

\begin{frame}{Motivation}
  \begin{itemize}
  \item Important to understand what makes programming languages tick
  \item Form an important layer of abstraction above hardware
  \item Lots of exciting work in the field
  \item Couldn't find a club/community on campus that specifically addressed PL and compiler enthusiasts
  \end{itemize}
\end{frame}

\begin{frame}{Exciting work in the field}
  \begin{exampleblock}{Formal Verification}
    \begin{itemize}
      \item Project Everest from Microsoft: Use F* to build formally verified computing stacks. Network protocols like QUIC, cryptography primitives
      \item MIT PLV: Koika \cite{koika2020}, fscq \cite{chen2015using}
    \end{itemize}
  \end{exampleblock}

  \begin{exampleblock}{Systems Programming}
    \begin{itemize}
    \item Modern languages like Rust and Zig aim to democratize systems programming and eliminate common-place bugs
    \item SSLab at Georgia Tech recently published RUDRA \cite{rudra2021}. Won distinguished artifact at SOSP'21
    \end{itemize}
  \end{exampleblock}
\end{frame}

\begin{frame}{Exciting work in the field (contd)}
    \begin{exampleblock}{ML frameworks}
    \begin{itemize}
    \item PyTorch, TensorFlow, and JAX use a Just-in-time (JIT) compiler to synthesize and run workloads
    \item Sam Gross authored an implementation of Python that gets rid of the GIL \cite{gross2021nogil}
    \item Development of typed array programming languages like Dex \cite{dexlang2021}
    \end{itemize}
  \end{exampleblock}
\end{frame}

\begin{frame}{What do we do?}
  \begin{itemize}
  \item Talks, workshops, and paper reading sessions
  \item Projects
  \item Networking and community building
  \end{itemize}
\end{frame}

\section{Why do we want to be a CoC org?}

\begin{frame}{Why CoC?}
  \begin{itemize}
  \item Target audience largely resides in the student body of the CoC
  \item CoC can help with organizing initiatives like a hackathon, sending students to conferences, and networking sessions
  \item Have a positive impact on the community
  \end{itemize}
\end{frame}

\section{Governance}

\begin{frame}{How goes it?}
  \begin{itemize}
  \item Elect a board to manage and run the club
  \item Prioritize the community over personal incentives
  \item Major decisions are driven through community deliberation
  \end{itemize}
\end{frame}

\begin{frame}{Board Structure}
  \begin{block}{Titles}
    \begin{itemize}
    \item Overall Director
    \item Director of Operations
    \item Director of Finance
    \item Director of Communication
    \end{itemize}
  \end{block}

  \begin{block}{Advisor}
    Dr. Qirun Zhang (\href{mailto://qrzhang@gatech.edu}{qrzhang@gatech.edu})
  \end{block}
\end{frame}

\begin{frame}{Code of Conduct}
  \begin{itemize}
  \item \url{https://dtyped-wiki.netlify.app/about/code-of-conduct/}
  \item Violations will be handled case-by-case
  \end{itemize}
\end{frame}

\section{State of the org and future plans}

\begin{frame}{State of the org}
  \begin{itemize}
  \item Active Discord community (over a 100 members)
  \item Hosted several talks and workshops (documented in the wiki)
  \item Attending conferences (Eg: LLVM dev conf)
  \end{itemize}
\end{frame}

\begin{frame}{Future plans (next semester)}
  \begin{itemize}
  \item Maintain a blog documenting some of our technical work
  \item Organize a langjam (we have a committee for it)
  \item Have a professor/engineer give a tech talk, and host a networking session thereafter
  \end{itemize}
\end{frame}

\begin{frame}{Future plans (long term)}
  \begin{itemize}
  \item Empower the next generation of innovators in the field
  \item Collaborate with research labs on campus to engage in novel research
  \item Develop and maintain a mainstream piece of OSS software
  \end{itemize}
\end{frame}

\section{Questions}

\begin{frame}
  \begin{itemize}
  \item Website: \url{https://dtyped.netlify.app}
  \item Wiki: \url{https://dtyped-wiki.netlify.app}
  \end{itemize}
\end{frame}

\begin{frame}[allowframebreaks]{Bibliography}
\printbibliography
\end{frame}

\end{document}
